\documentclass[]{article}
\usepackage{lmodern}
\usepackage{amssymb,amsmath}
\usepackage{ifxetex,ifluatex}
\usepackage{fixltx2e} % provides \textsubscript
\ifnum 0\ifxetex 1\fi\ifluatex 1\fi=0 % if pdftex
  \usepackage[T1]{fontenc}
  \usepackage[utf8]{inputenc}
\else % if luatex or xelatex
  \ifxetex
    \usepackage{mathspec}
  \else
    \usepackage{fontspec}
  \fi
  \defaultfontfeatures{Ligatures=TeX,Scale=MatchLowercase}
\fi
% use upquote if available, for straight quotes in verbatim environments
\IfFileExists{upquote.sty}{\usepackage{upquote}}{}
% use microtype if available
\IfFileExists{microtype.sty}{%
\usepackage{microtype}
\UseMicrotypeSet[protrusion]{basicmath} % disable protrusion for tt fonts
}{}
\usepackage[margin=1in]{geometry}
\usepackage{hyperref}
\PassOptionsToPackage{usenames,dvipsnames}{color} % color is loaded by hyperref
\hypersetup{unicode=true,
            pdftitle={Quantifying biological constraints on stream integrity for classification and priorization},
            pdfauthor={Marcus W. Beck (marcusb@sccwpr.org), Raphael D. Mazor (raphaelm@sccwrp.org), Scott Johnson (scott@aquaticbioassay.com), Phil Markle (phil@lacsd.org), Peter D. Ode (peter.ode@wildlife.ca.gov), Ryan Hill (hill.ryan@epa.gov), Eric D. Stein (erics@sccwrp.org)},
            colorlinks=true,
            linkcolor=Maroon,
            citecolor=Blue,
            urlcolor=blue,
            breaklinks=true}
\urlstyle{same}  % don't use monospace font for urls
\usepackage{graphicx,grffile}
\makeatletter
\def\maxwidth{\ifdim\Gin@nat@width>\linewidth\linewidth\else\Gin@nat@width\fi}
\def\maxheight{\ifdim\Gin@nat@height>\textheight\textheight\else\Gin@nat@height\fi}
\makeatother
% Scale images if necessary, so that they will not overflow the page
% margins by default, and it is still possible to overwrite the defaults
% using explicit options in \includegraphics[width, height, ...]{}
\setkeys{Gin}{width=\maxwidth,height=\maxheight,keepaspectratio}
\IfFileExists{parskip.sty}{%
\usepackage{parskip}
}{% else
\setlength{\parindent}{0pt}
\setlength{\parskip}{6pt plus 2pt minus 1pt}
}
\setlength{\emergencystretch}{3em}  % prevent overfull lines
\providecommand{\tightlist}{%
  \setlength{\itemsep}{0pt}\setlength{\parskip}{0pt}}
\setcounter{secnumdepth}{0}
% Redefines (sub)paragraphs to behave more like sections
\ifx\paragraph\undefined\else
\let\oldparagraph\paragraph
\renewcommand{\paragraph}[1]{\oldparagraph{#1}\mbox{}}
\fi
\ifx\subparagraph\undefined\else
\let\oldsubparagraph\subparagraph
\renewcommand{\subparagraph}[1]{\oldsubparagraph{#1}\mbox{}}
\fi

%%% Use protect on footnotes to avoid problems with footnotes in titles
\let\rmarkdownfootnote\footnote%
\def\footnote{\protect\rmarkdownfootnote}

%%% Change title format to be more compact
\usepackage{titling}

% Create subtitle command for use in maketitle
\newcommand{\subtitle}[1]{
  \posttitle{
    \begin{center}\large#1\end{center}
    }
}

\setlength{\droptitle}{-2em}
  \title{Quantifying biological constraints on stream integrity for
classification and priorization}
  \pretitle{\vspace{\droptitle}\centering\huge}
  \posttitle{\par}
  \author{Marcus W. Beck
(\href{mailto:marcusb@sccwpr.org}{\nolinkurl{marcusb@sccwpr.org}}),
Raphael D. Mazor
(\href{mailto:raphaelm@sccwrp.org}{\nolinkurl{raphaelm@sccwrp.org}}),
Scott Johnson
(\href{mailto:scott@aquaticbioassay.com}{\nolinkurl{scott@aquaticbioassay.com}}),
Phil Markle (\href{mailto:phil@lacsd.org}{\nolinkurl{phil@lacsd.org}}),
Peter D. Ode
(\href{mailto:peter.ode@wildlife.ca.gov}{\nolinkurl{peter.ode@wildlife.ca.gov}}),
Ryan Hill
(\href{mailto:hill.ryan@epa.gov}{\nolinkurl{hill.ryan@epa.gov}}), Eric
D. Stein (\href{mailto:erics@sccwrp.org}{\nolinkurl{erics@sccwrp.org}})}
  \preauthor{\centering\large\emph}
  \postauthor{\par}
  \date{}
  \predate{}\postdate{}

\usepackage{lineno}
\linenumbers
\usepackage{setspace}
\linespread{2}
\usepackage{cleveref}
\usepackage{acronym}
\acrodef{csci}[CSCI]{California Stream Condition Index}
\acrodef{nhd}[NHD]{National Hydrography Dataset Plus}

\begin{document}
\maketitle

\section{Introduction}\label{introduction}

\begin{itemize}
\item
  Degraded biological condition in streams can occur from individual or
  multiple stressors acting at different scales (Novotny et al.
  \protect\hyperlink{ref-Novotny05}{2005}; Townsend, Uhlmann, and
  Matthaei \protect\hyperlink{ref-Townsend08}{2008}; Leps et al.
  \protect\hyperlink{ref-Leps15}{2015}). Identifying and mitigating
  causes of poor condition requires an understanding of how stressors
  propogate across space and time. Incomplete knowledge on drivers of
  change or high level of uncertainty in how biology is linked to
  drivers can lead to ineffective management actions. Varying expenses
  and assurance of outcomes (e.g., varying costs and challenges of urban
  stream restoration (Kenney et al.
  \protect\hyperlink{ref-Kenney12}{2012}; Shoredits and Clayton
  \protect\hyperlink{ref-Shoredits13}{2013}))
\item
  Stream restoration relies extensively on identification and
  prioritization of sites where activities are expected to have desired
  outcomes. This requires an understanding of how stressors affect
  biological integrity to place bounds on reasonable expectations for
  what is likely to be a possible outcome of a management action. This
  requires identifying biological constraints or limits on the potential
  range of biological conditions. Identifying an appropriate context for
  observed conditions can be used to prioritize. Context can be defined
  by models, expert knowledge, and/or defined value sets.
\item
  We don't have good constraint tools to develop a context of
  expectation of what's possible at a site. This can help prioritize
  locations where management efforts will or will not have the intended
  outcomes. Biological filters act at different scales (Poff
  \protect\hyperlink{ref-Poff97}{1997}) and we can use this information
  to describe an expectation for prioritization that is scale-specific.
  Landscape-level constraints are particularly relevant for
  macroinvertebrate communities in streams (Sponseller, Benfield, and
  Valett \protect\hyperlink{ref-Sponseller01}{2001})
\item
  The goal of this study is to demonstrate application of a landscape
  model to classify and prioritize stream monitoring sites using
  estimated constraints on biological integrity. The model provides an
  estimate of context for biological condition that provides an
  expectation of what is likely to be achieved at a given site relative
  to large-scale drivers of stream health. The model was developed and
  applied to all stream reaches in California. A case study is used to
  demonstrate how the model can be used to classify and prioritize using
  guidance from a stakeholder group.
\end{itemize}

\section{Methods}\label{methods}

\subsection{Study area and data
sources}\label{study-area-and-data-sources}

\begin{itemize}
\item
  Brief description of CA, stream types and designated uses, PSA
  regions, management interests (e.g., southern vs.~northern CA)
\item
  Streamcat database used to quantify watershed land use at all sites
  (Hill et al. \protect\hyperlink{ref-Hill16}{2016})
\item
  Streamcat data linked to \ac{nhd} (USGS (US Geological Survey)
  \protect\hyperlink{ref-USGS14}{2014}), reach as individual unit for
  model output
\item
  \ac{csci} as measure of stream integrity (Ode et al.
  \protect\hyperlink{ref-Ode16}{2016}; Mazor et al.
  \protect\hyperlink{ref-Mazor16}{2016}), brief description of index
\end{itemize}

\subsection{Building and validating landscape
models}\label{building-and-validating-landscape-models}

A prediction model of the \ac{csci} was built to estimate likely ranges
of scores associated with land use gradients. Land use parameters were
urgan and agricultural land cover in the stream catchment (STREAMCAT).
The model is incomplete by design such that \ac{csci} scores were
modelled only in relation to landscape-level variables that are not
easily targeted by management. The model provided an explanation of
variation in scores related to constraints on biology and unexplained
variation was considered representative of additional, unmeasured
environmental variables that influence stream biointegrity. Maybe
describe modelling approach in (Mazor et al.
\protect\hyperlink{ref-Mazor16}{2016}) - which variables were used to
develop \ac{csci}.

Models were developed using quantile random forests to estimate a range
of likely \ac{csci} scores in different landscapes (Liaw and Wiener
\protect\hyperlink{ref-Liaw02}{2002}; D. R. Cutler et al.
\protect\hyperlink{ref-Cutler07}{2007}). The model predictions were used
to describe where bioassessment targets are unlikely to be met or where
streams are unlikely to be impacted. Calibration and validation data
were selected as xyz.

\subsection{Classifying streams and prioritizing
sites}\label{classifying-streams-and-prioritizing-sites}

\begin{itemize}
\item
  Description of SGRRMP and stakeholder group
\item
  Methods for estimating stream class - possibly/likely constrained,
  possibly/likely unconstrained, certainty and \ac{csci} threshold, some
  sites were unclassified
\end{itemize}

\textbf{Figure}

\begin{itemize}
\item
  Methods for estimating site performance - over, expected,
  underperforming, discussion of site types
\item
  Sensitivity analysis - how do classes, performance categories change
  with thresholds and certainty
\item
  Prioritization of types - stakeholder involvement
\end{itemize}

\section{Results}\label{results}

\subsection{State-wide patterns}\label{state-wide-patterns}

\begin{itemize}
\item
  Where does the model perform well, how does performance vary with
  validation and calibration datasets.
\item
  What is the consistency of patterns? For example, percent stream miles
  as xyz by PSA.
\end{itemize}

\emph{Figure} Statewide map.

\subsection{Case study}\label{case-study}

\begin{itemize}
\item
  San Gabriel River Regional Monitoring Program
\item
  Extent, classification, prioritization - probabilistic assessment to
  make broader conclusions.
\item
  Relationships with environmental variables for
  constrained/unconstrained locations. Maybe apply to
  hardened/non-hardened reaches in constrained locations.
\item
  What to do with unclassified streams - typical urban, typical ag.
\end{itemize}

\emph{Tables} Priority by type, by perspective

\section{Discussion}\label{discussion}

\begin{itemize}
\item
  What do priorities really mean? Depends on your interests, needs,
  values, etc.
\item
  Link with engineered channels study.
\end{itemize}

\section{Supplement}\label{supplement}

Online application.

\section*{References}\label{references}
\addcontentsline{toc}{section}{References}

\hypertarget{refs}{}
\hypertarget{ref-Cutler07}{}
Cutler, D. R., T. C. Edwards Jr., K. H. Beard, A. Cutler, K. T. Hess, J.
Gibson, and J. J. Lawler. 2007. ``Random Forests for Classification in
Ecology.'' \emph{Ecology} 88 (11): 2783--92.
doi:\href{https://doi.org/10.1890/07-0539.1}{10.1890/07-0539.1}.

\hypertarget{ref-Hill16}{}
Hill, R. A., M. H. Weber, S. G. Leibowitz, A. R. Olsen, and D. J.
Thornbrugh. 2016. ``The Stream-Catchment (StreamCat) Dataset: A Database
of Watershed Metrics for the Conterminous United States.'' \emph{Journal
of the American Water Resources Assocation} 52: 120--28.
doi:\href{https://doi.org/10.1111/1752-1688.12372}{10.1111/1752-1688.12372}.

\hypertarget{ref-Kenney12}{}
Kenney, M. A., P. R. Wilcock, B. F. Hobbs, N. E. Flores, and D. C.
Martínez. 2012. ``Is Urban Stream Restoration Worth It?'' \emph{Journal
of the American Water Resources Association} 48 (3): 603--15.
doi:\href{https://doi.org/10.1111/j.1752-1688.2011.00635.x}{10.1111/j.1752-1688.2011.00635.x}.

\hypertarget{ref-Leps15}{}
Leps, M., J. D. Tonkin, V. Dahm, P. Haase, and A. Sundermann. 2015.
``Disentangling Environmental Drivers of Benthic Invertebrate
Assemblages: The Role of Spatial Scale and Riverscape Heterogeneity in a
Multiple Stressor Environment.'' \emph{Science of the Total Environment}
536: 546--56.
doi:\href{https://doi.org/10.1016/j.scitotenv.2015.07.083}{10.1016/j.scitotenv.2015.07.083}.

\hypertarget{ref-Liaw02}{}
Liaw, Andy, and Matthew Wiener. 2002. ``Classification and Regression by
randomForest.'' \emph{R News} 2 (3): 18--22.
\url{http://CRAN.R-project.org/doc/Rnews/}.

\hypertarget{ref-Mazor16}{}
Mazor, R. D., A. C. Rehn, P. R. Ode, M. Engeln, K. C. Schiff, E. D.
Stein, D. J. Gillett, D. B. Herbst, and C. P. Hawkins. 2016.
``Bioassessment in Complex Environments: Designing an Index for
Consistent Meaning in Different Settings.'' \emph{Freshwater Science} 35
(1): 249--71.

\hypertarget{ref-Novotny05}{}
Novotny, V., A. Bartosová, N. O'Reilly, and T. Ehlinger. 2005.
``Unlocking the Relationship of Biotic Integrity of Impaired Waters to
Anthropogenic Stresses.'' \emph{Water Research} 39 (1): 184--98.
doi:\href{https://doi.org/10.1016/j.watres.2004.09.002}{10.1016/j.watres.2004.09.002}.

\hypertarget{ref-Ode16}{}
Ode, P. R., A. C. Rehn, R. D. Mazor, K. C. Schiff, E. D. Stein, J. T.
May, L. R. Brown, et al. 2016. ``Evaluating the Adequacy of a
Reference-Site Pool for Ecological Assessments in Environmentally
Complex Regions.'' \emph{Freshwater Science} 35 (1): 237--48.

\hypertarget{ref-Poff97}{}
Poff, N. L. 1997. ``Landscape Filters and Species Traits: Towards
Mechanistic Understanding and Prediction in Stream Ecology.''
\emph{Journal of the North American Benthological Society} 16 (2):
391--409.

\hypertarget{ref-Shoredits13}{}
Shoredits, A. S., and J. A. Clayton. 2013. ``Assessing the Practice and
Challenges of Stream Restoration in Urbanized Environments of the USA.''
\emph{Geography Compass} 7 (5): 358--72.
doi:\href{https://doi.org/10.1111/gec3.12039}{10.1111/gec3.12039}.

\hypertarget{ref-Sponseller01}{}
Sponseller, R. A., E. F. Benfield, and H. M. Valett. 2001.
``Relationships Between Land Use, Spatial Scale and Stream
Macroinvertebrate Communities.'' \emph{Freshwater Biology} 46 (10):
1409--24.
doi:\href{https://doi.org/10.1046/j.1365-2427.2001.00758.x}{10.1046/j.1365-2427.2001.00758.x}.

\hypertarget{ref-Townsend08}{}
Townsend, C. R., S. S. Uhlmann, and C. D. Matthaei. 2008. ``Individual
and Combined Responses of Stream Ecosystems to Multiple Stressors.''
\emph{Journal of Applied Ecology} 45 (6): 1810--9.
doi:\href{https://doi.org/10.1111/j.1365-2664.2008.01548.x}{10.1111/j.1365-2664.2008.01548.x}.

\hypertarget{ref-USGS14}{}
USGS (US Geological Survey). 2014. ``National Hydrography Dataset
available on the World Wide Web.''


\end{document}
